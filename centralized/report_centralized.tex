\documentclass[11pt]{article}

\usepackage{amsmath}
\usepackage{textcomp}

% Add other packages here %


% Put your group number and names in the author field %
\title{\bf Excercise 4\\ Implementing a centralized agent}
\author{Group \textnumero 4: Bruno \textsc{Wicht}, Adrian \textsc{Valente}}


% N.B.: The report should not be longer than 3 pages %


\begin{document}
\maketitle

\section{Solution Representation}

\subsection{Variables}
% Describe the variables used in your solution representation %
Let us first define a notion that we use in our solution: we call a TaskAction an action that is either a pickup or a delivery. A TaskAction is thus defined by its corresponding task, and a boolean specifying if it is a pickup or not. Obviously, if there are $N_T$ tasks there are $2N_T$ TaskActions.\\
We have used the following variables:
\begin{itemize}
\item An array $nextAction$ of $N_V+2N_T$ entries (one for each vehicle, and one for each TaskAction) mapping from a TaskAction to the next or to $NULL$, and from a vehicle to its first TaskAction.
\item An array $time$ of $2N_T$ entries mapping from each TaskAction to the position in which it is treated by its vehicle.
\item An array $vehicle$ of $2N_T$, mapping from each TaskAction to the vehicle that treats it.
\item Two arrays $deliver$ and $pickup$ of $N_T$ entries each mapping from a task to the TaskActions corresponding to its pickup and delivery.
\end{itemize}

\subsection{Constraints}
% Describe the constraints in your solution representation %
The constraints are the same as those specified in the slides, to which we add the following constraint (expressing the fact that a pickup should happen before a delivery, and that both actions must be done by the same vehicle):
\begin{equation}\label{constraint}
\begin{split}
& (a_j = deliver(t_i)) \land (a_k = pickup(t_i)) \\
& \Rightarrow (time(a_j) > time(a_k)) \land (vehicle(a_j) = vehicle(a_k))
\end{split}
\end{equation}

\subsection{Objective function}
% Describe the function that you optimize %
The function that we optimise is the total cost of a solution, that is the sum of the distances that each vehicle should travel during its plan:
\begin{equation*}
\begin{split}
C = &\sum_{i=1}^{2*N_T}(dist(a_i,nextAction(a_i))\cdot cost(vehicle(a_i)) \\
 &\quad + \sum_{i=1}^{N_V}(dist(initialCity(v_i),nextAction(v_i))\cdot cost(vehicle(a_i))
\end{split}
\end{equation*}


\section{Stochastic optimization}

\subsection{Initial solution}
% Describe how you generate the initial solution %
We have first tried to do as in the slides and give all tasks to the first vehicle for the inital solution. However, we found that it was not a good starting point, since it didn't lead to a very good solution. Instead, we start by a random solution: we randomly assign each task to a vehicle, and each vehicle carries the tasks that have been assigned to it in order (by delivering a task before picking up another).

\subsection{Generating neighbours}
% Describe how you generate neighbors %
In order to generate neighbors of a solution $sol$, we first randomly select a vehicle $v$ such that $nextAction(v) \neq NULL$, and then apply 2 primitives: $changeVehicle(sol,v)$ and $changeTaskOrder(sol,v)$.\\
The neighbors returned by $changeVehicle(sol,v)$ are all the solutions where one task carried by $v$ is given to another vehicle (which picks it up and delivers it before carrying out the rest of its plan).\\
The neighbors returned by $changeTaskOrder(sol,v)$ are all the consistent solutions where two actions in the plan of $v$ are permuted. The primitive tries out all permutations and returns only those who satisfy to the constraint \eqref{constraint}.

\subsection{Stochastic optimization algorithm}
% Describe your stochastic optimization algorithm %
Our stochastic optimization process follows the algorithm specified in the slides, the termination condition being that no improvement is seen among all neighbors generated (no neighbor with a smaller cost than the current cost is found). However, we found that this condition was rather harsh given that the neighbor generation is stochastic and can result in a low number of neighbors, especially when there are few tasks. As a consequence, we modified the termination condition so that the algorithm stops when no improvement is seen in at most $NMAX$ calls to $generateNeighbors$, $NMAX$ being a fixed parameter. 

\section{Results}

\subsection{Experiment 1: Model parameters}
% if your model has parameters, perform an experiment and analyze the results for different parameter values %

\subsubsection{Setting}
% Describe the settings of your experiment: topology, task configuration, number of tasks, number of vehicles, etc. %
% and the parameters you are analyzing %
We will be using the English topology, with 30 tasks and 4 vehicles (and random seeds 12345 and 54321). The only parameter that we will analyze is $NMAX$, which controls the termination condition of our algorithm.

\subsubsection{Observations}
% Describe the experimental results and the conclusions you inferred from these results %
Let us be blunt: the idea of calling $generateNeighbors$ several times does not bring much improvement: indeed, with $NMAX = 4$ we observed a total cost of $7234 \pm 955$ with $35 \pm 10$ iterations, while with $NMAX = 1$ we observed a total cost of $8359 \pm 1349$ and $17 \pm 6$. So there is an improvement if $NMAX$ is increased, but it is a very limited one, given that the computation time is doubled. That makes us think that the problem may not be of finding a minimum, but of not getting stuck in a local minimum, which appears to be the main factor for excessive costs.


\subsection{Experiment 2: Different configurations}
% Run simulations for different configurations of the environment (i.e. different tasks and number of vehicles) %

\subsubsection{Setting}
% Describe the settings of your experiment: topology, task configuration, number of tasks, number of vehicles, etc. %
We will try our method with the English topology, with 10, 30 and 50 tasks, and 4 and 2 vehicles. 

\subsubsection{Observations}
% Describe the experimental results and the conclusions you inferred from these results %
% Reflect on the fairness of the optimal plans. Observe that optimality requires some vehicles to do more work than others. %
% How does the complexity of your algorithm depend on the number of vehicles and various sizes of the task set? %
With 4 vehicles: we observe a typical cost of $3892 \pm 568$ with 10 tasks (computation time: $36 \pm 11$ ms), $7703 \pm 404$ with 30 tasks ($311 \pm 51$ ms) , and $18568 \pm 1277$ with 50 tasks ($538 \pm 133$ ms). There are huge disparities between the cost of each vehicle, but that is not necessarily a bad thing, since that can also be the case in the optimum. With 2 vehicles, let us simply say that the cost is a little higher.

\end{document}